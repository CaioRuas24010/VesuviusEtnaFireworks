% Codificação e idioma
\usepackage[utf8]{inputenc}
\usepackage[T1]{fontenc}
\usepackage[portuguese]{babel}

% Pacotes adicionais
\usepackage{wrapfig}
\usepackage[margin=1in]{geometry}
\usepackage{xcolor}
\pagecolor{yellow!10}
\definecolor{borda}{HTML}{971d21}
\usepackage{graphicx}
\graphicspath{{images/}}
\usepackage{anyfontsize}
\usepackage{tikz}
\usepackage{eso-pic}
\usepackage{titlesec}

% Configurações dos títulos
\titleformat{\chapter}[display]
  {\normalfont\bfseries\Huge\centering}
  {\chaptername\ \thechapter}
  {20pt}
  {\Huge}

% Redefinindo o formato do título das partes
\titleformat{\part}
  {\normalfont\Huge\bfseries\centering} % Formato da etiqueta e do título
  {\partname~\thepart}                  % Texto da etiqueta (ex: Parte I)
  {20pt}                                 % Espaçamento entre a etiqueta e o título
  {\Huge}                               % Formato aplicado ao título

% Ajustando o espaçamento antes e depois do título da parte
\titlespacing*{\chapter}
  {0pt}      % Espaço à esquerda
  {-50pt}    % Espaço antes do título (negativo para reduzir)
  {20pt}     % Espaço depois do título
% Definir a margem interna desejada

\newcommand{\bordamargem}{1cm}

\AddToShipoutPicture{%
  \begin{tikzpicture}[remember picture, overlay]
    % Desenhar a borda vermelha com margem interna
    \draw[borda, line width=1pt]
      ([xshift=\bordamargem, yshift=-\bordamargem]current page.north west) rectangle
      ([xshift=-\bordamargem, yshift=\bordamargem]current page.south east);
    % Adicionar pontos nas extremidades com margem interna
    \fill[borda] ([xshift=\bordamargem, yshift=-\bordamargem]current page.north west) circle (3pt);
    \fill[borda] ([xshift=-\bordamargem, yshift=-\bordamargem]current page.north east) circle (3pt);
    \fill[borda] ([xshift=\bordamargem, yshift=\bordamargem]current page.south west) circle (3pt);
    \fill[borda] ([xshift=-\bordamargem, yshift=\bordamargem]current page.south east) circle (3pt);
  \end{tikzpicture}
}